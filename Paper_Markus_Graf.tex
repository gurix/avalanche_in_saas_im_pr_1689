\documentclass[12pt, a4paper]{article}
\parindent0pt
\usepackage{fullpage}
\usepackage[utf8]{inputenc}
\usepackage[round]{natbib}
\usepackage{blindtext}
\usepackage{titlesec}
\usepackage{graphicx}
\bibliographystyle{plainnat}

\linespread{1.5}


\title{Were they able to know the danger? \\ The avalanche disaster from the 25th January 1689 in Saas im Prättigau.}
\author{Markus Graf}

\titleformat*{\section}{\normalfont\sffamily\bfseries}

\begin{document}
\maketitle
\abstract{\noindent The chronicles about the avalanche disaster from the 25th January 1689 don't cover any reasons that led to this event. Only recent researches revealed that the lack of protection forest in combination with heavy snowfalls was the cause. There are no direct  sources that can answer if they knew about the danger. But discussions about deforestation in that region and the knowledge about protection forest in other parts of Switzerland suggests that they were able to know the danger at least. }

\newpage
\setlength{\parindent}{0pt}

\section*{The avalanche disaster from the 25th January 1689}
\begin{quote}
\setlength{\parskip}{1.5em}
 It was the day named  conversion of St. Paul, at the 25 of January 1689 at 8 o'clock, when we had to feel wrath of the highest of all God, as two dry avalanches went over the community of Saas, and in which 59 People lost their lives. From the community of Saas 48 died, 5 from Conters, two from Fideris, 3 from Davos and one from Küblis.
 
The first avalanche, which went off at 8 o'clock, started outwards at Calmur in Büel, called the ``Hannen''; from that point on it took the way through the forest and through the mountain down till inside Marthels; from there to the Landquartstutz. Cruelly it tore down   forest, stables, houses, arboretums. 10 houses were destroyed, 15 People died; several were dug out alive. Two girls and an older woman for example, the woman was buried with first avalanche, the girl with the second. They were dug out not until the next day. 

After the first avalanche,  about 9 o'clock,  a good amount of friends  and neighbours were running for help because of the miserable screams and the raging church bells, its damage was not visible from Conters because of the wind and weather. They were also running from Küblis and till 12 o'clock they dug out People and livestock. At this hour another avalanche arrived. It started at Calandagrat and went down the mountain across Drumalinis, over the Barglein at Gruob, across the Landwasser, with the result that on the other side of the valley was covered with wood, 	household effects and that sort of thing. This avalanche also tore down everything in front of it. 12 houses  were destroyed, 44 People rest death or died from the consequences. Most of the rescue team was beaten down while escaping; people escaped inwards got away. Several begun to work again, others fled. At this hour a sort of people from Serneus went to help. We dug out alive five persons from a pile on youth Rudolfs Brosis house. We wish a cheerful resurrection to those who passed away to God, the Lord and a blessed the end the survivors. Many were dug out alive, many were healthy, several were injured, those I wish patience and get well. But those who lost all their belongings, those console oneself with Hiobs words of the first chapter in verse 21: And said, Naked came I out of my mother's womb, and naked shall I return thither: the lord gave, and the lord had taken away; blessed be the name of the lord. They buried all corpses within 9 weeks, most of them were intact as they died in their beds. 

After the second avalanche several people worked till midnight and dug out people death and alive at all times. They buried first 23 bodies in the churchyard of Saas, 3 in Conters and one in Küblis. The other time 12 in Saas, and the third time 6. 

I, Daniel Jost from Conters, witnessed the events with my own eyes. I recognized the second avalanche and tried to warn them to escape inwards by shouting and whistle imploringly. Half of the people fled inwards where knocked but stayed alive, the others who fled outwards all died. I fled, was frightened and scared, I can't tell more of this event.
 -- Daniel Jost, witness of the tragedy \footnote{Translated from \citet[p.~50-51]{hansemann1995saaser}}
\end{quote}

\setlength{\parskip}{1.5em}

More then 300 years later, nothing reminds on the tragedy Daniel Jost witnessed in this winter. The area where the avalanches went down is now cluttered with new houses, a good portion of them are secondary homes hosting tourists. Weather forecasts, risk assessments, forestation, avalanche barrier and emergency plans makes it safe to live there. Because of a strong winter, an evacuation of the area Raschnal was necessary in 2009, but luckily nothing happened. Today people know about the dangers and authorities know how to deal with this threats. 

The report form Jost is leaving behind a feeling of sorrow and raises the question whether they knew about the danger or was it coming out of the blue? One of the interesting matter is that none of the reports questioned the reason or the circumstances of that event. Jost was just speaking of the wrath of God. None of the reports that is available in the Saaser hometown book\footnote{\citet[p.~50-54]{hansemann1995saaser}} is discussing the cause of this event. 

\section*{Learning from the past -  Reconstruction of the event from 1689 reveals the reason behind the tragedy}
A hazard assessment including a map of risk was done for Saas im Prättigau in 2004\footnote{\cite{teufen2004}}. They analysed  the event in January 1689 to answer the question whether such a disaster is still possible today.  Based on the report from the chronicles of Jost, estimation off the type, the dimensions and the track of the avalanches was made. It was concluded that an avalanche with the same properties would not reach the same force again today. The difference between now and then was the the forestation along the track of the avalanche. The absence of protection forest was the only conceivable reason how the avalanche built such a massive force. 

\begin{figure}[htpb!]
\includegraphics[width=\textwidth,natwidth=610,natheight=642]{literature/gefahrenkarte.jpg}
\caption{Map of risk for Saas im Prättigau \citep{gefahrenkarte}}
\end{figure}

The theory about the lack of forest is also supported by a myth around this event, in which a child-cradle was blown over the river to the other side of the valley whereas the child stayed uninjured.\footnote{\citet[p.~53]{hansemann1995saaser}} The track of the avalanche was described as following:  ``The mountain slope above the village takes about two hour steep upwards and consists of beautiful meadows interrupted by small strips of forest.''

Not only the topographical properties were decisive factors, to build such an intense avalanche, it also needs a good amount of snow that triggers the drift. The weather conditions were also reported in the myth\footnote{\citet[p.~53]{hansemann1995saaser}}, it tells that it was snowing for two weeks before the event. The same day, another avalanche went of in St.Antönien, a community nearby. The event was written down in the Ruosch chronicles, and it confirms that snowfalls were significant in this region, Finze-Michaelsen describes it as catastrophic Winter\footnote{\citet[p.~18]{finze1988geschichte}}.  
 
\section*{The history of protection forest in Switzerland}
People are save today especially because of forestation that protects the people below the slope. To discuss if people from Saas im Prättigau knew the danger we first have to review the history of protection forest in Switzerland.

Today a protection forest is by definition an area that is forest to protect against avalanches, rockfall, mudflow, or flooding. The term varies in its meaning, especially the German term ``Bannwald''. The historical meaning of ``Bann'' is an area that is controlled by a landlord where nobody is allowed to use it without his permission. In medieval times the Bannwald was a protected forest and only in  modern times the word is used as forest that protects people from natural disasters.

One of the first known letter that prohibits the use of the forest to protect people in Switzerland was found in Andermatt from the 1397\footnote{\citet[p.~104]{pfister2002tag}}. Every use of wood, also deadwood was prohibited, people knew that the forest prevents avalanches from going off. Protected forests were sometimes also guarded by local forest and field warden, but the reason behind this was mostly just to protect the property not the to obtain the protective function. 

But simple people had no other chance as clearing the forest in higher areas. They needed the wood for constructing and heating, they simply had no alternatives. So the forest was looted anyway, especially in the 17th century. Only in the 19th century people started to question this actions, especially because a huge amount of floods. Until then forest protection concerned only the cantons. Strong political pressure leads 1874 to a new section in the federal constitution that authorizes federal government to manage Swiss forest centrally. In 1876 all mountain side forests were protected and reforestation was subsidised. 26 years later every forest in Switzerland was protected by the forest police act\footnote{\citet[p.~108]{pfister2002tag}}.

\section*{Discussion about knowledge of protection forest in Saas im Prättigau}
The history of protection forest in Switzerland reveals that there were regions where people knew about the importance forest to protect them from harm. There are other hints that people knew about the issue of deforestation in this region. 

The parish letter of Saas im Prättigau around 1550 reveals, that there were already strict rules concerning the cutting of timber. Hansenmann is speculating that one reason for this regulation was the smelting works that needed a lot of wood. He also states that a lot of challenges the community is facing now were already present at that time, especially avalanches and mudflow. But there is still no  explicit reason.\footnote{\citet[p.~27, 139]{hansemann1995saaser}}.

Regarding the sustainability of the forest another interesting quote can be found in \citet[p.~7]{finze1988geschichte}, in which 	chronicler Georg Engle 1807 is writing that people from St.Antönien are complaining 300 years ago about deforestation and waste of wood that is no longer available for their offspring. 

Putting everything together, there is no explicit source that describes that people of Saas im Prättigau knew about the danger, especially in combination with this hard winter and the lack of forest. There is also no source that describes any actions right after the tragic event of 1689. But we certainly know that deforestation regarding the protection of ownership and sustainability was a known issue. We also know that knowledge off the consequences from deforestation regarding natural hazards was existing. From our perspective, such a transfer of knowledge was possible, especially because of the fact that Saas im Prättigau was and still is an important passage that connects Klosters and Davos with the Rhine Valley. It is possible that documents that governs actions after the avalanche regarding protection of the forest existed but get lost. One explanation could be the devastating fire of 1735 that destroyed the village especially the center with it's church and probably ruined some government documents \footnote{\citet[p.~55]{hansemann1995saaser}}. 

\section*{Perspectives of the future}
The importence of protection against natural hazards was gaining importance with better life quality and progresses of civilization and technology. It's unlikely that knowledge about the importance of the protection against avalanches in this region will be lost. But we learned that knowledge and memories of such a tragic event can fade away. It is important to conserve such events and keep in mind that such dangers existed an can reappear.

\bibliography{literature}
\end{document}